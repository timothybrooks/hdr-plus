\documentclass[12pt]{exam}
\usepackage{amsmath}
\usepackage[left=1in, right=1in, top=1in, bottom=1in]{geometry}
\usepackage{graphicx}
\usepackage{ mathrsfs }
\usepackage{relsize}
\usepackage{amsmath}
\usepackage{hyperref} %\url{https://...}

\graphicspath{{/}}

\newcommand{\answerbox}[1]{
\begin{framed}
\hspace{5.65in}
\vspace{#1}
\end{framed}}

\newcommand{\textbox}[1]{\noindent\fbox{\parbox{\textwidth}{#1}}}

%%%%%%%%%%%%%%%%%%%%%%%%%%%%%%
%%%%% FILL OUT INFO HERE %%%%%
%%%%%%%%%%%%%%%%%%%%%%%%%%%%%%

\newcommand{\myname}{Tim Brooks, Suhaas Reddy}
\newcommand{\coursenumber}{15-769 }
\newcommand{\duedate}{21 November 2016}
\newcommand{\andrewid}{tebrooks, suhaasr}
\newcommand{\collaborators}{None}
\newcommand{\professor}{Kayvon Fatahalian}
\newcommand{\assignmentnumber}{Checkpoint}
\newcommand{\sol}[1]{\leavevmode \begin{solution} #1 \end{solution}}


%%%%%%%%%%%%%%%%%%%%%%%%%%%%%%
%%%%%%%%%%%%%%%%%%%%%%%%%%%%%%
%%%%%%%%%%%%%%%%%%%%%%%%%%%%%%

\pagestyle{head}

\headrule \header{\textbf{\coursenumber  \assignmentnumber}}{\myname}{\textbf{Page \thepage}}

\pointsinmargin \printanswers

\setlength\answerlinelength{2in} \setlength\answerskip{0.3in}

\begin{document}
\addpoints
\begin{center}
\textbf{\large{\coursenumber
\\  \vspace{0.2in} \assignmentnumber
}}

 \vspace{0.2in}

 \large{Due: \duedate}
\end{center}

\vspace{0.5in}

\text{Name: \myname}

\vspace{0.2in}

\text{Andrew IDs: \andrewid}

\vspace{.2in}

\text{Professor: \professor}

\

\vspace{0.2in}

\newpage

\begin{questions}
\question{Starter code (if any)}

    \sol{
    We are not using any starter code for the main HDR+ pipeline, but for now we are using dcraw to convert DNG RAW images into a greyscale format which we will treat as RAW for the rest of the pipeline. Dcraw is implemented as a command line script rather than a library, so we can't easily call it directly from the pipeline, but we may customize it more in the future.
    }

\question{Initial workloads/datasets for project. Why are these workloads a good choice?}

    \sol{
    We have taken a few handheld raw bursts with a DSLR camera as well as a regularly exposed comparison shot for each burst. We would prefer to use the original Google dataset from the burst photography paper, but Andrew Adams suggested we take our own photographs while we wait for the dataset to be released. We took a wide array of photographs including a pathological case (a flag flapping in the wind) to make sure our pipeline correctly handles various lighting and misalignment conditions. In this pathological case it should detect the extreme misalignment and reject the tiles containing the flag in all alternate frames.
    }

\question{Restated expectations}

    \sol{
    We hope to replicate the results of the HDR+ burst photography paper to better understand Halide, computational photography, and, more specifically, how to build a more accurate reconstruction of a ``true" signal by composing multiple noisy samples. We believe that this kind of technique represents the future of imaging (e.g. the light camera \url{https://light.co/camera}) and that a successful implementation of this pipeline would demonstrate our deep understanding of the subject.

    This project represents a significant challenge because alignment, merging, and tone mapping all require complicated algorithms with differing data access patterns (and therefore differing schedules). We aim to produce results comparable to the HDR+ paper in a reasonable amount of time, which could be challenging given that the original team consisted of experts in the field.

    We hope to demonstrate our success with both subjective results showing success and failure cases of our algorithm, and possibly by running a live demo of the pipeline during our presentation to the class. Additionally, we will include graphs and charts comparing our performance to the HDR+ Burst implementation and Photoshop's HDR.
    }
    
\question{Schedule}
    \sol{
    
    Dec 16th:
    
        \hspace{.2 in} Wrap pipeline with bare interface that allows users to input a burst of photos and see the results. Produce comparisons between our results, Google's results, Photoshop's HDR, and spatial averaging. Test performance on some subset of the Google image database and create presentation/analysis. Practice presentation. Ideally practice a live demo with photos taken at the beginning of the presentation (if the pipeline runs quickly enough).
    
    Dec 9th:
    
        \hspace{.2 in} Finish implementing and scheduling tone mapping.
    
    Dec 2nd: 

        \hspace{.2 in} Schedule merging and coarse implementation of tone mapping.
    
    Nov 28th:

        \hspace{.2 in} Implement merging and schedule alignment

    Nov 21st: 

        \hspace{.2 in} Finish implementing alignment

    }
    
\end{questions}
\end{document}