\documentclass[12pt]{exam}
\usepackage{amsmath}
\usepackage[left=1in, right=1in, top=1in, bottom=1in]{geometry}
\usepackage{graphicx}
\usepackage{ mathrsfs }
\usepackage{relsize}
\usepackage{amsmath}
\usepackage{hyperref} %\url{https://...}

\graphicspath{{/}}

\newcommand{\answerbox}[1]{
\begin{framed}
\hspace{5.65in}
\vspace{#1}
\end{framed}}

\newcommand{\textbox}[1]{\noindent\fbox{\parbox{\textwidth}{#1}}}

%%%%%%%%%%%%%%%%%%%%%%%%%%%%%%
%%%%% FILL OUT INFO HERE %%%%%
%%%%%%%%%%%%%%%%%%%%%%%%%%%%%%

%doot

\newcommand{\myname}{Suhaas Reddy, Tim Brooks}
\newcommand{\coursenumber}{15-769 }
\newcommand{\duedate}{21 November 2016}
\newcommand{\andrewid}{suhaasr}
\newcommand{\collaborators}{None}
\newcommand{\professor}{Kayvon Fatahalian}
\newcommand{\assignmentnumber}{Checkpoint}
\newcommand{\sol}[1]{\leavevmode \begin{solution} #1 \end{solution}}


%%%%%%%%%%%%%%%%%%%%%%%%%%%%%%
%%%%%%%%%%%%%%%%%%%%%%%%%%%%%%
%%%%%%%%%%%%%%%%%%%%%%%%%%%%%%

\pagestyle{head}

\headrule \header{\textbf{\coursenumber  \assignmentnumber}}{\myname}{\textbf{Page \thepage}}

\pointsinmargin \printanswers

\setlength\answerlinelength{2in} \setlength\answerskip{0.3in}

\begin{document}
\addpoints
\begin{center}
\textbf{\large{\coursenumber
\\  \vspace{0.2in} \assignmentnumber
}}

 \vspace{0.2in}

 \large{Due: \duedate}
\end{center}

\vspace{0.5in}

\text{Name: \myname}

\vspace{0.2in}

\text{Andrew ID: \andrewid}

\vspace{.2in}

\text{Professor: \professor}

% End title box
\

\vspace{0.2in}

\newpage

\begin{questions}
\question{Starter code (if any)}

    \sol{
    We are not using any starter code for our implementation of the HDR+ paper.
    }

\question{Initial workloads/datasets for project. Why are these workloads a good choice?}

    \sol{
    
    }

\question{Restated expectations}

    \sol{
    We hope to show that ... The reason why this project is challenging is ... We will demonstrate success/failure by running ~this experiment~ and showing the following demo.
    }
    
\question{Schedule}
    \sol{
    
    Dec 16th:
    
        \hspace{.2 in} Wrap pipeline with a bare interface that allows users to input a burst of photos and see the results. Produce comparisons between our results, Google's results, Photoshop's HDR, and spatial averaging. Test performance on some subset of the Google image database and create presentation/analysis. Practice presentation. Ideally practice a live demo with photos taken at the beginning of the presentation (if the pipeline runs quickly enough).
    
    Dec 9th:
    
        \hspace{.2 in} Finish implementing and scheduling tone mapping.
    
    Dec 2nd: 

        \hspace{.2 in} Schedule merging and coarse implementation of tone mapping.
    
    Nov 28th:

        \hspace{.2 in} Implement merging and schedule alignment

    Nov 21st: 

        \hspace{.2 in} Finish implementing alignment
        
    }
    
\end{questions}
\end{document}